\documentclass[aspectratio=1610,14pt]{beamer}
\usepackage{sty/style}
%
%
%%% Begin:

\title{%
    kyasual : The modern\\ beamer theme
}
\subtitle{
    -smart casual beamer theme-
}
\author{%
    1XXXXX\ Sample Author
}
\institute[Sample Univ.]{%
    Sample Univ. 
}
\date{%
    \today
}

\begin{document}
%
%
\begin{frame}
\maketitle 
\end{frame}

\begin{frame}[fragile]{Abstract}
    なんか良い感じのbeamer theme †\alert{kyasual}†ができた!
\end{frame}

\section{Introduction}

\begin{frame}
    \tableofcontents[currentsection]
\end{frame}

\begin{frame}{Introduction}
    \begin{textblock}{目的}
        \begin{itemize}
        \item{従来のbeamer templateは可愛げがない.}
        \thusitem{なんか\alert{可愛げのある}themeを書く.}
        \end{itemize}
    \end{textblock}
    \begin{textblock}{要請}
        \begin{enumerate}
            \item{\alert{可愛げがある.}}
            \item{\textbf{エンジニア向けLT}に使える.}
            \item{でも\textbf{数学のゼミ}でも使えるくらいには\alert{フォーマル}.}
            \thusitem{ギリcasualじゃないエセcasual,即ち \alert{kyasual}.}
                \begin{itemize}
                    \annoitem{世の中にはsmart casualという言葉がある.}
                \end{itemize}
        \end{enumerate}
    \end{textblock}
\end{frame}

\begin{frame}{How to Setup}
    \begin{textblock}{Preparation}
        \begin{itemize}
            \item{\textbf{Fork} this repository.}
            \item{install \textbf{LaTeX} and \textbf{latexmk}.}
        \end{itemize}
    \end{textblock}

    \begin{textblock}{Build}
        \begin{enumerate}
            \item{\lstinline|mv sampleslide.tex (your slide title)|}
            \item{\lstinline|make|}
                \thusitem{(your slide title).pdf should be generated.}
        \end{enumerate}
    \end{textblock}
\end{frame}

\section{Samples}
\begin{frame}
    \tableofcontents[currentsection]
\end{frame}

\begin{frame}{Overview}
    \begin{textblock}{This is Sample section.}
        \begin{enumerate}
            \item{text}
            \item{itemize}
            \item{block and box}
            \item{code}
            \item{math}
            \item{image} 
            \item{tree}
        \end{enumerate}
    \end{textblock}
\end{frame}

\begin{frame}{text sample}
    This is Sample Page.\\
    You can make \\
    \alert{new line}\\
    whenever you want to do.\\
    You can typing about this match. If a character overflows, a new line is inserted by itself.\\
    If you want to enphasys word, you can use \textbf{bold}, \alert{alert}, or \textit{italic}. \\
    font :
    \begin{itemize}
        \item{Main: GenshinGothic}
        \item{Italic: \textit{TimesItalic}}
    \end{itemize}
    ref example:\cite{bib1,bib2}
\end{frame}

\begin{frame}{日本語サンプル}
    これは日本語のサンプルページです.\\
    好きなタイミングで\\
    \alert{改行}\\
    できます.\\
    一行あたりはだいたいこのくらい書けて,余ったら勝手に改行します.\\
    強調は,\textbf{太字}と\alert{アラート}があります.\\
    フォント:源真ゴシック
    \\
    参考文献例:\cite{bib1,bib2}
\end{frame}

\begin{frame}{itemize sample}
    \begin{itemize}
        \item{普通のitem}
            \okitem{良い例のitem}
            \ngitem{ダメな例のitem}
            \begin{itemize}
                \annoitem{ダメな例の正しい例のitem}
            \end{itemize}
            \thusitem{従って,良いitemizeが書ける.}
            \butitem{逆に言うと,良いitemizeしか書けない.}
    \end{itemize}
    \begin{enumerate}
        \item{enumerateもできる.}
        \item{subitemは}
            \begin{enumerate}
                \item{こんな感じ.}
            \end{enumerate}
    \end{enumerate}
\end{frame}

\begin{frame}{block and box sample}
    \begin{block}{block}
        block
    \end{block}
    \begin{exampleblock}{example}
        example
    \end{exampleblock}
    \begin{alertblock}{alert}
        alert
    \end{alertblock}
    \begin{simplebox}
        simple box
    \end{simplebox}
\end{frame}

\begin{frame}[fragile]{code sample}
    \begin{columns}
        \begin{column}{0.5\textwidth}
            \begin{lstlisting}[language=Haskell]
class Monad m where 
    (>>=) :: m a -> (a -> m b) -> m b 
    return :: a -> m a

instance Monad (King k) where
    f >>= m = State $ \s ->
        let (k', a) = runState f k
            in runState (m a) k'
            \end{lstlisting}
        \end{column}
        \begin{column}{0.5\textwidth}
            モナドの王,モナ王 \\
            \textbf{KING MONAD}\\
            \lstinline|inline|
        \end{column}
    \end{columns}
\end{frame}


\begin{frame}{math sample1}
    \begin{block}{Def. 1 Sample}
        任意の集合$A$と,$A$上で定義された二項関係$\to_{\alpha}$の和$R = \bigcup_{a \in I} \to_a $の対$(A, R)$をSampleと呼ぶ.
    \end{block}

\end{frame}

\begin{frame}{math sample 2}
  \begin{equation}
    x = a_0 + \cfrac{1}{a_1 
      + \cfrac{1}{a_2 
        + \cfrac{1}{a_3 + \cfrac{1}{a_4} } } }
  \end{equation}

  \[
  \sqrt[n]{1+x+x^2+x^3+\dots+x^n}
  \]
\end{frame}

\begin{frame}{tree sample}
    \scriptsize
    \begin{prooftree}
        \AxiomC{}
        \RightLabel{(true)}
        \UnaryInfC{$\vdash \mathrm{true} : \mathrm{bool}$}
        \AxiomC{}
        \RightLabel{(int)}
        \UnaryInfC{$\vdash 0 : \mathrm{int}$}
        \AxiomC{}
        \RightLabel{(int)}
        \UnaryInfC{$\vdash 1 : \mathrm{int}$}
        \RightLabel{(tpl)}
        \BinaryInfC{$\vdash \langle 0,1 \rangle  : \mathrm{int\times int}$}
        \RightLabel{(fst)}
        \UnaryInfC{$\vdash \mathrm{fst}\langle 0,1 \rangle :\mathrm{int}$}
        \RightLabel{(tpl)}
        \BinaryInfC{$\vdash \langle \mathrm{true  ,fst} \langle 0,1 \rangle \rangle : \mathrm{bool \times int}$}
        \AxiomC{}
        \RightLabel{(var)}
        \UnaryInfC{$x:\mathrm{bool\times int} \vdash x : \mathrm{bool \times int}$}
        \RightLabel{(snd)}
        \UnaryInfC{$x:\mathrm{bool\times int} \vdash \mathrm{snd}\:x :\mathrm{int}$}
        \RightLabel{(let)}
        \BinaryInfC{$\vdash \mathrm{let}\: x = \langle \mathrm{true  ,fst} \langle 0,1 \rangle \rangle \: \mathrm{in \: snd}\:x:\mathrm{int}$}
    \end{prooftree}
    \normalsize
\end{frame}

\begin{frame}{image sample}
    \begin{center}
        \includegraphics[width = 5cm]{./fig/kya}
    \end{center}
\end{frame}

\section{How to Use?}

\begin{frame}
    \tableofcontents[currentsection]
\end{frame}

\begin{frame}{How to Use?}
    \begin{textblock}{基本的には,普通のbeamerと同じです.}
        \begin{itemize}
            \item{または,これとsampleslide.texを眺めてください.}
            \item{要望があれば,ドキュメントを追加します.}
                \begin{itemize}
                    \item{要望は,issueか@kyawawayへお願いします.}
                \end{itemize}
        \end{itemize}
    \end{textblock}
\end{frame}

\begin{frame}[fragile]{Features1: textbox}
    \begin{columns}
        \begin{column}{0.5\textwidth}
            \begin{lstlisting}[language=TeX]
% using like other blocks

\begin{textblock}{textblock title1}
    textblock body1
\end{textblock}

\begin{textblock}{textblock title2}
    \begin{itemize}
        \item{textblock body2}
        \item{\alert{itemize}との併用で}
        \okitem{いい感じ}
    \end{itemize}
\end{textblock}
            \end{lstlisting}
        \end{column}
        \begin{column}{0.5\textwidth}
            \begin{textblock}{textblock title1}
                textblock body1
            \end{textblock}
            \begin{textblock}{textblock title2}
                \begin{itemize}
                    \item{textblock body2}
                    \item{\alert{itemize}との併用で}
                    \okitem{いい感じ}
                \end{itemize}
            \end{textblock}
        \end{column}
    \end{columns}
\end{frame}


\begin{frame}[fragile]{Features2: itemize icon}
    \begin{columns}
        \begin{column}{0.5\textwidth}
            \begin{lstlisting}[language=TeX]
% using feature command 
% in itemize environment
\begin{itemize}
    \okitem{ok}
    \negitem{neg}
        \begin{itemize}
            \thusitem{thus}
            \butitem{but}
        \end{itemize}
        \annoitem{annotate}
\end{itemize}

\begin{enumerate}
    \item{enum1}
    \item{enum2}
        \begin{enumerate}
            \item{enum2.1}
        \end{enumerate}
\end{enumerate}            
            \end{lstlisting}
        \end{column}
        \begin{column}{0.5\textwidth}
            \begin{itemize}
                \okitem{ok}
                \ngitem{neg}
                    \begin{itemize}
                        \thusitem{thus}
                        \butitem{but}
                    \end{itemize}
                \annoitem{annotate}
            \end{itemize}

            \begin{enumerate}
                \item{enum1}
                \item{enum2}
                    \begin{enumerate}
                        \item{enum2.1}
                    \end{enumerate}
            \end{enumerate}
        \vspace{2em}
        \end{column}
    \end{columns}
\end{frame}

\begin{frame}[fragile]{Features3: simplebox}
    \begin{columns}
        \begin{column}{0.5\textwidth}
            \begin{lstlisting}[language=TeX]
\begin{simplebox}
    simple box
\end{simplebox}

\begin{simplebox}
    beamercolorboxで適宜設定してもらうことは\\
    あんまり想定していない.
\end{simplebox}
            \end{lstlisting}
        \end{column}
        \begin{column}{0.5\textwidth}
            \begin{simplebox}
                simple box
            \end{simplebox}
            \begin{simplebox}
                beamercolorboxで適宜設定してもらうことは\\
                あんまり想定していない.
            \end{simplebox}
        \end{column}
    \end{columns}
\end{frame}

\section{Conclusion}

\begin{frame}
    \tableofcontents[currentsection]
\end{frame}

\begin{frame}{Conclusion}
    \begin{textblock}{ふつうにめっちゃカジュアルになった.}
        \begin{itemize}
            \okitem{でも\alert{いい感じ}なので満足:)}
        \end{itemize}
    \end{textblock}
\end{frame}

\begin{frame}[allowframebreaks]{Reference}
    \printbibliography
\end{frame}
%
%
%
%
\end{document}

%%% End
