\documentclass[aspectratio=1610,14pt]{beamer}
\usepackage{sty/style}

%
%
%%% Begin:

\title{%
    Sample Title \\
    サンプルタイトル
}
\subtitle{
    -sample subtitle-
}
\author{%
    1XXXXX\ Sample Author
}
\institute[Sample Univ.]{%
    Sample Univ. 
}
\date{%
    \today
}

\begin{document}
%
%
\begin{frame}
\maketitle 
\end{frame}
\begin{frame}[fragile]{abstract}
This is sample page.\\[4pt] 
これはサンプルページです.
\end{frame}

\section{section1}
\subsection{subsection1}

\begin{frame}
    \tableofcontents[currentsection]
\end{frame}

\begin{frame}[fragile]{code sample}
    \begin{columns}
        \begin{column}{0.5\textwidth}
            \begin{lstlisting}
class Monad m where 
    (>>=) :: m a -> (a -> m b) -> m b 
    return :: a -> m a
            \end{lstlisting}
        \end{column}
        \begin{column}{0.5\textwidth}
            バーニングモナド \\
            \textbf{BURNING MONAD}
        \end{column}
    \end{columns}
\end{frame}

\begin{frame}{itemize sample}
    itemize
    \begin{itemize}
        \item \textbf{bold item}
        \begin{itemize}
            \item sub item
        \end{itemize}
    \end{itemize}
\end{frame}

\begin{frame}{math sample}
    \begin{block}{\textit{Def. 1} Sample}
        任意の集合$A$と,$A$上で定義された二項関係$\to_{\alpha}$の和$R = \bigcup_{a \in I} \to_a $の対$(A, R)$をSampleと呼ぶ.
    \end{block}
\end{frame}

\begin{frame}{block sample}
    \begin{block}{block}
        block
    \end{block}
    \begin{alertblock}{alert}
        alert
    \end{alertblock}
    \begin{exampleblock}{example}
        example
    \end{exampleblock}
\end{frame}

\begin{frame}{日本語サンプル}
    これは日本語のサンプルページです.
\end{frame}

\section{section2}

\begin{frame}
    \tableofcontents[currentsection]
\end{frame}
%
\section{section3}

\begin{frame}
    \tableofcontents[currentsection]
\end{frame}
%
%
%
%
\end{document}

%%% End
